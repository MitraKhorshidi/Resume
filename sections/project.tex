%-------------------------------------------------------------------------------
%	SECTION TITLE
%-------------------------------------------------------------------------------
\cvsection{Projekte}


%-------------------------------------------------------------------------------
%	CONTENT
%-------------------------------------------------------------------------------
\begin{cventries}

%---------------------------------------------------------
\cventry
{Ein schlankes und schnelles Online-Menü für sein Café und Restaurant} % subtitle
{Allegro BookStore} % title
{2022} % Date
{Shiraz} % Location
{
  \mitalt{Stack:}
  NextJS \mitdiv ReactJS \mitdiv TailwindCSS \mitdiv Adobe XD \mitdiv Git \mitdiv Cloudflare
  % \newline
  % Die Website wurde mit Adobe XD entworfen, um die Präferenzen des Unternehmens für UI-Inhalt und Styling zu erfüllen.
  % Um eine schnelle, mobilfreundliche Website zu entwickeln, benutze ich NextJS, um die statische Website-Generierung zu ermöglichen
  % und TailwindCSS, um den Mobile-First- und Utility-First-Ansatz zu verfolgen.
}

%---------------------------------------------------------
\cventry
{Dynamisches Reservierungs- und Essensbestellsystem für ein Restaurant} % subtitle
{Boomi Restaurant} % title
{2022} % Date
{Shiraz} % Location
{
  \mitalt{Stack:}
  NextJS \mitdiv ReactJS \mitdiv TailwindCSS \mitdiv Adobe XD \mitdiv Git \mitdiv Vercel
  % \newline
  % User Stories, UML-Modelle, Datenbanktabellen und UI-Prototypen wurden
  % nach Software-Engineering-Methoden entwickelt. Dann verwende ich NextJS als Framework für die Entwicklung von UI
  % und Backend und verbinde sie mit der Datenbank durch Sequelize.
}

%---------------------------------------------------------
\cventry
{Persönliches Portfolio, das meine Fähigkeiten und Kenntnisse aufzeigt und untersucht} % subtitle
{iMitra} % title
{2021} % Date
{Shiraz} % Location
{
  \mitalt{Stack:}
  NextJS \mitdiv ReactJS \mitdiv CSS3 \mitdiv Adobe XD \mitdiv Git \mitdiv Cloudflare
  % \newline
  % Dieses Portfolio wurde im Laufe der Zeit umgebaut, umgestaltet und erweitert.
  % Zuerst wurde mit WordPress Elementor gebaut und dann mit NextJS, ReactJS und CSS3 umgebaut.
  % Um sicherzustellen, dass das Projekt gut dokumentiert war und um eine klare Aufzeichnung des Entwicklungsprozesses zu erhalten
  % Ich habe alle Änderungen in Git gespeichert.
}
% 
% %---------------------------------------------------------
% \cventry
% {Analyzing information management system's concepts} % subtitle
% {Boomi restaurant (phase 2)} % title
% {2021} % Date
% {Shiraz} % Location
% {
%   \mitalt{Stack:}
%   Visual Paradigm
%   \newline
%   The second phase centered around importance of data in modern industry competition.
%   Containing how to make a manager accompany with an information system while also ensuring
%   the security of these valuable assets.
% }
% 
% %---------------------------------------------------------
% \cventry
% {Analyzing eBusiness concepts} % subtitle
% {Boomi restaurant (phase 1)} % title
% {2021} % Date
% {Shiraz} % Location
% {
%   \mitalt{Stack:}
%   Visual Paradigm
%   \newline
%   The overall objective of the first phase was to offer an e-business system to a local company
%   in order to help them stay competitive in their industry. So focused on analyzing e-business concepts
%   includes who are customers and what are the values represented to them, extracting customers' and employers' needs,
%   specifying the service and marketing.
% }
% 
% %---------------------------------------------------------
% \cventry
% {Personal Portfolio} % subtitle
% {iMitra (Old website)} % title
% {2020} % Date
% {Shiraz} % Location
% {
%   \mitalt{Stack:}
%   WordPress \mitdiv Elementor \mitdiv CSS3
% }

%---------------------------------------------------------
\cventry
{Online-Shop für eine lokale Schneidermarke in Shiraz} % subtitle
{Fatima Fashion Tailoring} % title
{2020} % Date
{Shiraz} % Location
{
  \mitalt{Stack:}
  WordPress \mitdiv Elementor \mitdiv CSS3 \mitdiv Adobe XD \mitdiv WooCommerce
  % \newline
  % Ich entwarf UI mit der Marke Styling Thema und Vorlieben, 
  % und benutzte WordPress Elementor, um ihnen alles zu geben, was ein Online Shop braucht, 
  % von der Einführung und dem Kontakt mit den Kunden bis zum Verkauf und der Bestandsverwaltung.
}

%---------------------------------------------------------
% \cventry
% {subtitle} % subtitle
% {title} % title
% {2021} % Date
% {Shiraz} % Location
% {
%   \mitalt{Stack:}
%   XXX \mitdiv YYY 
%   \newline
%   DESC
% }

% ---------------------------------------------------------
\end{cventries}
